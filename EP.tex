\documentclass[compress, blue]{beamer}
\usetheme{Warsaw}
\usecolortheme{orchid}
\beamersetuncovermixins{\opaqueness<1>{25}}{\opaqueness<2->{15}}

\usepackage{listings}

% German
\usepackage[ngerman]{babel}

% Read input file as UTF-8
\usepackage[utf8]{inputenc}
% Font encoding: T1 for european languages

\usepackage[T1]{fontenc}
% Textcomp for 128 more characters (including EUR)
\usepackage{textcomp}

% Latin Modern instead of Computer Modern
\usepackage{lmodern}

\usepackage{graphicx}

\title{Extrem Programming}
\subtitle{Gründe, Methoden und Prozesse}
\author{Nick Zbinden}
\date{\today}

\begin{document}

\begin{frame}
  \titlepage
\end{frame}

\begin{frame}\frametitle{Extrem Programming?}
  \includegraphics{sky.jpeg}
\end{frame}

\begin{frame}
  \includegraphics[width=10cm]{lan.jpg}
\end{frame}

\begin{frame}
  \begin{quote}
    Extrem Programming ist der Einzug von Wahrheit und Vernunft in die Software Entwicklung.
  \end{quote}
  Pavel Maier
\end{frame}

\begin{frame}\frametitle{Probleme}
  \includegraphics[width=10cm]{ep.jpg}  
\end{frame}

\begin{frame}\frametitle{Probleme}
  \begin{itemize}
  \item Software wird nicht rechtzeitig fertig. \pause
  \item Entwicklung wird fürhzeitig beendet. \pause
  \item Änderungsaufwand wird zu gross. \pause
  \item Software wird wegen Unzuverlässigkeit nicht genutzt \pause
  \item Kunde weiss nicht was er will! \pause
  \item Kunde ändert seine Meinung.
  \end{itemize}
\end{frame}

\begin{frame}\frametitle{Ängste}
  \begin{itemize}
  \item Kunde \pause
    \begin{itemize}
    \item Kriegt nicht was er will \pause
    \item Zahlt zu viel \pause
    \item Pläne sind Märchen.  \pause
    \item Kann Anforderungen nicht ändern. 
    \end{itemize}
  \end{itemize}
\end{frame}

\begin{frame}\frametitle{Ängste}
  \begin{itemize}
  \item Entwickler \pause
    \begin{itemize}
    \item Kunde verlangt zu viel. \pause
    \item Verantwortung aber keine Autorität. \pause
    \item Qualität Opfern für Termine. \pause
    \end{itemize}
  \end{itemize}
\end{frame}

\begin{frame}
  \includegraphics[width=10cm]{werte.png}
\end{frame}

\begin{frame}\frametitle{Best Practises}
  \begin{itemize}
  \item Best Practieses \pause
    \begin{itemize}
    \item Keine Code-Besitzer\pause
    \item Refactoring \pause
    \item Keine Überstunden \pause
    \item Small releases
    \end{itemize}
  \end{itemize}
\end{frame}

\begin{frame}\frametitle{Pair Programming}
    \includegraphics[width=5cm]{pair.jpg}
\end{frame}

\begin{frame}\frametitle{Pair Programming}
  \begin{itemize}
  \item Keine Know How Monople (Bus-Faktor) \pause
  \item Kürzere Einareitungszeiten \pause
  \item Weniger Ablenkung beim Programmieren \pause
  \item \textit{Besserer Code} \pause
  \item Studien
  \end{itemize}
\end{frame}

\begin{frame}\frametitle{Test Driven Development}
  \begin{itemize}
  \item Jede Funktion wird getestet.  \pause
  \item Test wird zuerst geschrieben. \pause
  \item Beim Einchecken wird alles getestet \pause
  \item Kompleze System einfach ändern\pause
  \item Performence tests, Securety tests, Scalierungs tests
  \end{itemize}
\end{frame}

\begin{frame}\frametitle{Rollen}
  \begin{tabular}{l l}
    \hline
    Rollen & Aufgaben \\
    \hline
    Kunde & Entscheidet, was gemacht wird, gibt Rückmeldung \\
    Entwickler &  Entwickelt das Produkt \\
    Projektmanager & Führung des Teams \\
    Benutzer & Wird das zu erstellende Produkt nutzen \\
    \hline
  \end{tabular}
\end{frame}

\begin{frame}\frametitle{Prozesse}
  \includegraphics[width=8cm]{loop2.jpg}
\end{frame}

\begin{frame}\frametitle{Release Plan}
  \begin{itemize}
  \item alle 2-3 Monate
  \item User Storys
    \begin{itemize}
    \item Karten erstellen
    \item Aufwand
    \item Risiko
    \item usw.
    \end{itemize}
  \end{itemize}
\end{frame}

\begin{frame}\frametitle{User Storys}
    \includegraphics[width=8cm]{userstorys.jpg}
\end{frame}

\begin{frame}\frametitle{User Storys}
    \includegraphics[width=8cm]{cards.jpg}
\end{frame}

\begin{frame}\frametitle{Iterations Plan}
  \begin{itemize}
  \item alle 1-3 Wochen
  \item Es wird bestimmt welech Users Storys gemacht werden
  \item User Storys werden in Engineerings Tasks unterteilt.
    \begin{itemize}
    \item Technische Anweisung
    \end{itemize}
  \item Kunde kann jederzeit Storys anpassen, neue erstellen oder
    wegwerfen.
  \item Statistiken über letzte Iteration. Velocity!
  \end{itemize}
\end{frame}

\begin{frame}\frametitle{Engineering Tasks}
    \includegraphics[width=7cm]{etask.jpg}
\end{frame}

\begin{frame}\frametitle{Standup-Meeting}
  \begin{itemize}
  \item Jeden Morgen
  \item Erzählen was man gemacht hat. Probleme!
  \item Wie lange hat man noch an seinem Task.
  \item Engineerings Tasks werden verteilt.
    \begin{itemize}
      \item Jeder Schätzt wie lange er dafür braucht.
    \end{itemize}
  \end{itemize}
\end{frame}

\begin{frame}\frametitle{Standup Meeting}
    \includegraphics[width=10cm]{standup.jpg}
\end{frame}


\begin{frame}\frametitle{Gesamtübersicht}
    \includegraphics[width=7cm]{diagramm.png}
\end{frame}

\begin{frame}
  The End
\end{frame}

\end{document}
